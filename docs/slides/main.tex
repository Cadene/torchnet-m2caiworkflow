\documentclass[compress,xcolor=table,11pt]{beamer}

% Packages
\usepackage[english, french]{babel}
\usepackage[utf8]{inputenc}
\usepackage[T1]{fontenc}
\usepackage{datetime}
\usepackage{longtable}
\usepackage{booktabs}
\usepackage{wrapfig}
\usepackage{array}
\usepackage{setspace}

\usepackage{float} % add option H for table
\usepackage{subcaption} % add subfigure and subcaption, see package options

% Possible options:
%  - nosectionpages: no pages between sections
%  - flama: use flama font, requires xelatex/lualatex to compile
%  - compressminiframes: put the heading list bullets indications pages on 1 line
\iffalse
\usepackage{fontspec}
\setmainfont[
    Path           = font/,
    Extension      = .otf,
    Ligatures      = TeX,
    BoldFont       = Flama-Bold,
    ItalicFont     = Flama-BasicItalic,
    BoldItalicFont = Flama-BoldItalic
]{Flama-Light}
\fi
%\usetheme[nosectionpages, flama]{upmc}
\usetheme[nosectionpages]{upmc}

\definecolor{hsrmSec3Comp}{rgb}{1,0.509803922,0}
\definecolor{hsrmSec3CompDark}{rgb}{0.666666667,0.333333333,0}

% Title page
\title{M2CAI Workflow Challenge 2016}
\foottitle{M2CAI Workflow Challenge 2016} % optional, by default same as title
\subtitle{Fine tuning CNN with HMM smoothing} % optional
\date{21th October 2016} %\formatdate{4}{10}{2016}}
\author{Rémi \textsc{Cadène}, Thomas \textsc{Robert}, Nicolas \textsc{Thome}, Matthieu \textsc{Cord}}
\institute{University Pierre and Marie Curie - LIP6 - MLIA}




% Biblatex
\usepackage[backend=bibtex,
style=authoryear,
citestyle=authoryear]{biblatex}
\bibliography{library.bib}
\AtEveryBibitem{%
	\clearfield{urldate}%
	\clearfield{urlday}%
	\clearfield{urlmonth}%
	\clearfield{urlyear}%
	\clearfield{doi}%
	\clearfield{issn}%
	\clearfield{isbn}%
	\clearfield{pagetotal}%
	\clearfield{url}%
	\clearfield{timestamp}%
}
\AtEveryCitekey{\UseBibitemHook}
\renewcommand*{\bibfont}{\footnotesize}
\let\oldcite\cite
\renewcommand{\cite}[1]{[\oldcite{#1}]}

\def\tightlist{}


\begin{document}


\begin{frame}[plain]
	\titlepage
	\setcounter{framenumber}{0}
\end{frame}

\section{Context} \subsection{}\label{}

\begin{frame}{M2CAI Workflow Dataset}
	
		\begin{figure}
			\centering
			\includegraphics[width=.45\linewidth]{images/m2cai.jpg}
			%		\caption{80000 train images, 20000 test images, 101 classes}
			\label{fig:2images}
		\end{figure}
	
	Videos resolution is 1920 x 1080, shot at 25 frames per second at the IRCAD research center in Strasbourg, France.
	
	\begin{itemize}
		\item 27 training videos %(67,595 images)
		%\item 22 train videos %(59,493 images)
		%\item 5 val videos %(8,062 images)
		\item 15 test videos %(28,732 images)
		%\item 8 classes (CleaningCoagulation, CalotTriangleDissection, CLippingCutting, etc.)
	\end{itemize}
	
\end{frame}

\begin{frame}{M2CAI Workflow Dataset}

	1 of 8 classes for each frames:
	\begin{itemize}
		\item TrocarPlacement
		\item Preparation
		\item CalotTriangleDissection
       	\item ClippingCutting
       	\item GallbladderDissection
       	\item GallbladderPackaging
       	\item CleaningCoagulation
       	\item GallbladderRetraction
    \end{itemize}

\end{frame}

\begin{frame}{M2CAI Workflow Goal and Measure}
	
	  Online prediction: $P(y | x_i, x_{i-1}, x_{i-2}, ...)$  
		
		$x_i$:= frame $i$, and $y$:= classes
		
		%Detecting at which of the 8 phases of the operation each frames belong.
		
		
		Useful to:
		\begin{itemize}
			\item monitor surgeons
			\item trigger automatic actions
		\end{itemize}
		
		
		Measures:
		- Jaccard similarity coefficient:
	    $J(A,B) = \frac{| A \cap B |}{| A \cup B|} = \frac{| A \cap B |}{| A| + |B| - |A \cap B|}$
	    
	  - Accuracy top1: nb frames well classified / nb total frames
		
\end{frame}




\section{Our method} \subsection{}\label{}

\begin{frame}{Two fold}

	Training models to classify from images (frames)
	
	Extract features from CNN
	Fine tuning CNN
	
	Smoothing the predictions of our model
	
	1. averaging
	2. HMM
	
\end{frame}


\begin{frame}{1. Extracting images}

	Train
	
	Val

\end{frame}

\begin{frame}{2. Training a frame classifier}

	Features extraction
	

\end{frame}

\begin{frame}{2. Training a frame classifier}

	Fine tuning CNN
	

\end{frame}

\begin{frame}{2. Training a frame classifier}

	Fine tuning CNN with Weldon
	

\end{frame}

\begin{frame}{3. Smoothing the predictions}

	Avereging
	

\end{frame}

\begin{frame}{3. Smoothing the predictions}

	HMM online
	

\end{frame}

\begin{frame}{3. Smoothing the predictions}

	HMM offline
	

\end{frame}
	
	
\section{Experiments} \subsection{}\label{}

\begin{frame}{Validation set}
	

\end{frame}

\begin{frame}{Visualization}

	by classes	
	
\end{frame}

\begin{frame}{Visualization}

	hmm A
	
\end{frame}

\begin{frame}{Visualization}

	hmm mus
	
\end{frame}


\section{Conclusion} \subsection{}\label{}

\begin{frame}{Conclusion}

	lolz
	
\end{frame}

\end{document}
